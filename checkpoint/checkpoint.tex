\documentclass[paper=a4, fontsize=11pt]{article} % A4 paper and 11pt font size
\usepackage[bottom=1.2in]{geometry}

\usepackage{amsmath,amsfonts,amsthm} % Math packages

\usepackage{graphicx}

%----------------------------------------------------------------------------------------
% TITLE SECTION
%----------------------------------------------------------------------------------------

\title{	Simulation of Pedestrian Movement Outside Football Stadium}

\author{Xiong Ding, **, ** (add your name)} % Your name

\date{\today} % Today's date or a custom date


%----------------------------------------------------------------------------------------
% content section
%----------------------------------------------------------------------------------------
\begin{document}

\maketitle % Print the title

\section{Problem Description}

\section{Literature review}

\section{Conceptual model}

\subsection{Inputs}

In order to simulate the pedestrian flow outside Georgia Tech's Bobby Dodd Stadium,
we need some real physical data about the stadium, pedestrians and street configuration
around the stadium. Also, variables like pedestrians' speed, space needed by people 
with different body sizes should vary and follow some distribution. 

\paragraph{Velocity} Regarding the difference of age, gender, and so on, pedestrian
will not have a uniform velocity. Follow reference ** and our owe daily experience, 
we propose that pedestrians' velocity is described by a Gaussian distribution of form
\begin{equation}
  \label{eq:velocity}
  v = \exp\{\frac{(x-1.34)^2}{0.26^2}\}
\end{equation}
In cellular automata, this means that some pedestrian may move more than one

\paragraph{Space of each pedestrian} We know that different people may

\subsection{Outputs}

\subsection{simulation components and rules}

\paragraph{Map construction}

\paragraph{Floor model}

\paragraph{Pedestrian model}

\paragraph{update strategy}

%------------------------------------------------------------------------

\end{document}